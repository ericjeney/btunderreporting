\section{Introduction}

BitTorrent is a successful decentralized system for transferring files. It allows numerous peers to cooperatively download files from a server that is otherwise underpowered. The fairness of this system depends on peers being honest with eachother, though this isn't always the case.

BitTorrent peers can gain an advantage over other peers by forgoing complete honesty and instead strategically revealing their pieces. In particular, by underreporting the selection of pieces that they have, peers can remain interesting for longer than they usually would. A small number of underreporting peers allows those peers to gain an unfair advantage over their honest counterparts, while a large number or underreporters degrades the network and can result in slower speeds for everyone.

However, while it is trivial to determine when a peer is overreporting, it is much more difficult to detect underreporting. When a peer is overreporting, you may request that it send you a piece that it doesn't have and it will have to inform you that it cannot fill your request. After this situation arises some number of times, it becomes safe to assume that a peer is overreporting and cease trading with them. Due to the difficulty in proving that a peer has a piece that it claims it does not, underreporting has no such method of easy detection.

Due to the difficulties involved in identifying underreporting peers, we design two general methods to disincentivize underreporting. In one method, our client prioritizes the peers that it shares with based on the number of pieces that each peer has and it do not. In the other it assumes that after a peer becomes uninteresting it takes some amount of time for honest peers to become interesting again. Based on this, it waits for that amount of time before allowing itself to become again interested in an uninteresting peer.

Our insight is that by removing the incentive to underreport it will no longer be necessary to identify underreporters. It is not necessary to identify and punish the peers that are underreporting if it is possible to disincentivize them from underreporting in the first place.

The remainder of this paper is laid out as follows. In Section 2 we present related work. Section 3 contains our first proposed disincentivizing solution, based on the number of pieces that one peer has and our client does not. We discuss our other solution, based on waiting before becoming interested in an uninteresting peer, in Section 4. Finally, we conclude in Section 5.

% Taken from what Zach sent. We'll need to expand it a bit to match Neil's expected format.
% In a BitTorrent network, a user has incentive to send selective HAVE messages to other peers. While normally a HAVE message for a piece of a file is sent to all peers when the user acquires that piece, a strategic under-reporter (as we are calling him/her) will only send out HAVE messages for the more common pieces it has, and not report its rarer pieces. By doing so, peers of the strategic under-reporter will maintain interest in the strategic under-reporter, because it can keep revealing pieces it has (that few others have) one-by-one. The peers will be less interested in trading with each other, since they received the same pieces of the file from the strategic under-reporter. Levin et al. showed that such a strategy improves the download speed of the under-reporter by prolonging interest, but that the average download time of all peers in the network increased by about 12\% when all peers adopted this strategy$^{[1]}$.