\subsection{Buffer Uninteresting Peers}

Place uninteresting peers in a \textbf{timeout}, collecting their new HAVE messages in a buffer to be examined at the end of the timeout.

As users receive new pieces, they send HAVE messages out to their peers who in turn send back an \textit{interested} message if they do not have the piece. As long as a user is missing at least one piece from a peer, that peer is considered \textbf{interesting}.

Under-reporters take advantage of this by minimizing the amount of pieces they report to their peers at a time. When a peer becomes uninterested in them, they send out another HAVE message, so the peer becomes interested in them again.

When a peer sends back an \textit{interested} message to the under-reporter, the under-reporter stops sending HAVE messages. Thus, if the peer were to wait before sending back the \textit{interested} message, they would receive more HAVE messages from under-reporters.

This will disincentivize under-reporting because it will slow down the under-reporter due to constantly being put in a timeout. In addition, his peers will receive more HAVE messages, making the under-reporting strategy grow closer to normal reporting.

\textbf{Downsides:} If the timeout is too short, under-reporters would hardly be effected, so the problem would not be solved. Conversely, if it is too long, all users will be slowed down from the wait.

If the buffer is too small, HAVE messages could go unreceived, resulting in lost trades. The buffer being too large would have memory issues for a large number of peers, but each user only has a small number of peers, so this should not be a problem.

The main time that there would be a problem for normal reporters is when they have almost all the pieces, and many other people become uninteresting.
