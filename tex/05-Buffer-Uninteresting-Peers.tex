Another possible solution to the underreporting problem is to generate interest based on \textit{have} messages. Peers send \textit{interested} messages to any peers which have pieces that they do not. Likewise, peers send \textit{uninterested} messages to any peers which do not have any pieces that they do not. Peers relay which pieces they receive through \textit{have} messages.

We make three main assumptions about the behaviors of honest and dishonest peers within a BitTorrent system.

Let $Q$ be a peer of swarm $S$ who is employing the underreporting strategy presented by Levin et. al \cite{dlbittorrent}. Let $A$ and $B$ be some other peers of swarm $S$ who are honest (they are not employing the underreporting strategy). Let $R$ be an integer number of rounds. Assume the following:

$$ \mathrm{Uninterested}(A, Q) \longleftrightarrow \mathrm{Have}(Q, A) $$
$$ \mathrm{Uninterested}(A, B) \longrightarrow \mathrm{Wait_Interesting}(B, R) $$
$$ \mathrm{Uninterested}(A, Q) \longrightarrow \mathrm{Wait_Interesting}(Q, 0) $$\newline

$Q$ will send a \textit{have} message to $A$ if and only if $A$ sends $Q$ an \textit{uninterested} message. If $A$ sends and \textit{uninterested} message to $B$, it will take some number of rounds, $R$, before $B$ regains $A$'s interest. If $A$ sends an \textit{uninterested} message to $Q$, it will take zero rounds before $Q$ regains $A$'s interest.

These assumptions are the foundations for interest-based disincentivizing. In an interest-based system, peers will use the assumptions above in order to create a system in which the best strategy is honest piece revelation.

\subsection{Buffer Uninteresting Peers}

Underreporters rely on \textit{interested} messages from honest peers in order to determine when they can stop sending selective \textit{have} messages. By virtue of this fact, we come up with the following strategy.

Place uninteresting peers in a \textbf{timeout}, collecting their new \textit{have} messages in a buffer to be examined at the end of the timeout.

By buffering the \textit{have} messages we receive, we can effectively eliminate any information from getting to the underreporting peers. They must then assume that they have not successfully re-gained our interest, and continue to send out \textit{have} messages. If we wait for long enough, the underreporting peer will send us all the possible \textit{have} messages. This forces them into honesty.

The optimal strategy for an underreporter in this case is to report honestly, because then they spend the minimal amount of time in \textbf{timeout}. It is important to note that this does not benefit the system unless a majority of peers employ the buffering strategy. The underreporter can still successfully underreport to all the other honest peers in the swarm.
