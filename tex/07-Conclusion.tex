\section{Conclusion}

We have generated two solutions to the problem of strategic piece revelation (underreporting) in BitTorrent. While it is feasible that a solution may exist which identifies underreporting peers and punishes them, we have reached the conclusion that identifying underreporters without peer collusion is harder than disincentivizing underreporting through other means. We formalize two algorithms, Bitfield Differences and Round-Based Interest, and outline the advantages and disadvantages of each solution. Both algorithms create an environment in which the incentive to underreport is nullified, eliminating the need to identify underreporters. Each algorithm requires no modification of the protocol, or existing implementation. Furthermore, neither algorithm makes assumptions about client implementations of other peers in a swarm. These two characteristics make these algorithms theoretically ideal for wide-spread deployment, making them a practical means of defending against underreporting. The Bitfield Differences algorithm exploits the unchoking algorithm as a means of disincentivizing underreporters. It uses the bitfields of other peers in order to weight the auction scheme, thus rewarding peers who have an abundance of pieces. The Round-Based Interest algorithm exploits BitTorrent interest and enforces a system in which interest is only re-evaluated based on expectations of an honest peer's behavior.

There is plenty of future work to be done. A controlled and live-cluster analysis needs to be run on the algorithms proposed in this paper. There are theoretical limitations to both the Bitfield Differences and Round-Based Interest algorithms. That being said, it is possible (and also likely) that some of these theoretical limitations will be fine in practice. Conversely, it may also be the case that we have made assumptions which are fine in theory, but present a practical disadvantage. Our research also chose to neglect the possibility of an identify-and-punish method to eliminate underreporting. If a reliable method of identification were formalized, it would be of both theoretical and practical importance. That result would imply that both overreporting and underreporting are not feasible within the constraints of the BitTorrent protocol. Lastly, there exist other questions within the realm of the BitTorrent protocol which are closely related to BitTorrent underreporting. One such question is, how do we encourage seeders to remain in and contribute to a swarm? The issue of seeder promotion is another open question which has practical implications for the success and efficiency of the BitTorrent protocol.
